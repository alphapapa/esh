%%%%%%%%%%%%%%%%%%%%%%%%%%%
% DO NOT MODIFY THIS FILE %
%%%%%%%%%%%%%%%%%%%%%%%%%%%

%% Instead, redefine the relevant macros in your own source file, **before**
%% including this file.  This file is overwritten every time you run esh2tex.

%%%%%%%%%%%%%%
%% Core ESH %%
%%%%%%%%%%%%%%

% Packages
%%%%%%%%%%

\RequirePackage{xcolor} % TeXLive-recommended
\RequirePackage[normalem]{ulem} % TeXLive-recommended

\RequirePackage{relsize} % TeXLive-latex-extra
\renewcommand{\RSsmallest}{1pt}
\renewcommand{\RSlargest}{50pt}

% Fonts
%%%%%%%

% \ESHFont is for code blocks
\providecommand*{\ESHFont}{\ttfamily}

% \ESHInlineFont is for inline code samples
\providecommand*{\ESHInlineFont}{\ESHFont}

% \ESHFallbackFont is applied to characters not covered by \ESHFont
\providecommand*{\ESHFallbackFont}{\ESHFont}

% Utils
%%%%%%%

% \ESHNoHyphens disables hyphenation
\providecommand*{\ESHNoHyphens}{\hyphenpenalty=10000}

% \ESHBlockDash is a dash disallowing line breaks
\providecommand*{\ESHBlockDash}{\hbox{-}\nobreak}

% \ESHCenterInWidthOf{#A}{#B} prints #B centered in a box as large as #A.
\newdimen\ESHtempdim%
\providecommand*{\ESHCenterInWidthOf}[2]
  {\settowidth\ESHtempdim{#1}%
   \makebox[\ESHtempdim][c]{#2}}

\RequirePackage{iftex}
\ifXeTeX
  % \ESHWithFallback{#A} prints #A in the current font if possible, falling back to \ESHFallbackFont
  % Adapted from https://tug.org/pipermail/xetex/2011-November/022319.html
  \def\ESHWithFallback#1{%
    \begingroup
      \def\found{#1}%
      \def\notfound{\ESHFallbackFont#1}%
      \ifnum\XeTeXfonttype\font>0%
        \ifnum\XeTeXcharglyph`#1>0\found\else\notfound\fi
      \else
        \setbox0=\hbox{\tracinglostchars=0\kern1sp#1\expandafter}%
        \ifnum\lastkern=1\notfound\else\found\fi
      \fi
    \endgroup}
\else
  % Fall back to just using \ESHFallbackFont
  \def\ESHWithFallback#1{\ESHFallbackFont#1}
\fi

% \ESHInlineSpecialChar and \ESHBlockSpecialChar are used by ESH to indicate
% non-ascii characters, which may need a fallback font.
\DeclareRobustCommand*{\ESHInlineSpecialChar}[1]
  {{\ESHFont\ESHWithFallback{#1}}}
\DeclareRobustCommand*{\ESHBlockSpecialChar}[1]
  {{\ESHCenterInWidthOf{\ESHFont{a}}{\ESHInlineSpecialChar{#1}}}}

% \ESHInlineRaise and \ESHBlockRaise implement sub/superscripts
\DeclareRobustCommand*{\ESHInlineRaise}[2]
  {\raisebox{#1}{\relsize{-2}#2}}
\DeclareRobustCommand*{\ESHBlockRaise}[2]
  {\rlap{\ESHInlineRaise{#1}{#2}}\hphantom{#2}}

\makeatletter
% ESHUnderline produces a straight underline
\DeclareRobustCommand*{\ESHUnderline}[1]
  {\bgroup\markoverwith{#1\rule[-0.5ex]{2pt}{0.4pt}}\ULon}
% ESHUnderwave produces a wavy underline
\DeclareRobustCommand*{\ESHUnderwave}[1]
  {\bgroup\markoverwith{#1\lower3.5\p@\hbox{\ESHSmallWaveFont\char58}}\ULon}
\font\ESHSmallWaveFont=lasyb10 scaled 400
\makeatother

% Environments
%%%%%%%%%%%%%%

% \ESHInlineBasicSetup is used by \ESHInline
\providecommand*{\ESHInlineBasicSetup}{\ESHNoHyphens\ESHInlineFont}

% \ESHInline is used for inline code
% Note the extra pair of braces in the definition
\providecommand*{\ESHInline}[1]{{\ESHInlineBasicSetup#1}}

% \ESHBlockBasicSetup is used by \ESHBlock
\providecommand*{\ESHBlockBasicSetup}
  {\setlength{\parindent}{0pt}\setlength{\parskip}{0pt}
   \ESHNoHyphens\raggedright\ESHFont\spaceskip=\fontdimen2\font\xspaceskip=0pt}

\makeatletter
% \ESHSkip is the amount to skip before and after an ESHBlock
\@ifundefined{ESHSkip}{\newlength{\ESHSkip}\setlength{\ESHSkip}{\baselineskip}}{}

% \ESHBlock is used for code blocks
\@ifundefined{ESHBlock}
  {\newenvironment{ESHBlock}
     {\par\ESHBlockBasicSetup\addvspace{\ESHSkip}}
     {\par\addvspace{\ESHSkip}}}{}

% \ESHBoxSep is the vertical padding of \ESHFBox
\@ifundefined{ESHBoxSep}{\newlength{\ESHBoxSep}\setlength{\ESHBoxSep}{1pt}}{}

% \ESHFBox{#color}{#lineWidth}{#contents} wraps #contents in an border of width
% #lineWidth and of color #color.  The box has no horizontal padding, its
% vertical padding is determined by \ESHBoxSep, and it doesn't affect the height
% of the current line.
\newdimen\ESHBoxTempDim%
\providecommand*{\ESHBox}[3]
  {\setlength{\fboxsep}{\ESHBoxSep}%
   \setlength{\fboxrule}{#2}%
   \setlength{\ESHBoxTempDim}{\dimexpr-\fboxsep-\fboxrule\relax}%
   \vphantom{#3}\smash{\fbox{\hspace*{\ESHBoxTempDim}#3\hspace*{\ESHBoxTempDim}}}}
\makeatother

%%%%%%%%%%%%%%%%%%%%%%%%%%%%%%%%%%%%%%%%%%%
%% \ESHpv: ESH Precomputed-Verbs support %%
%%%%%%%%%%%%%%%%%%%%%%%%%%%%%%%%%%%%%%%%%%%

%% Adapted from https://tex.stackexchange.com/questions/336837/

%% Utilities

% \ESHpvEnterVerbMode sets all specials to catcode 12
% * \@makeother sets the catcode of its argument to 12 (other)
% * \dospecials applies \do to each special character
% (see http://tex.stackexchange.com/questions/12721/control-command-arguments)
\makeatletter
\def\ESHpvEnterVerbMode{\let\do\@makeother\dospecials}
\makeatother

\def\ESHpvNotFoundMessage#1{No highlighting found for "#1"; falling back to "texttt".}
% \ESHpvNotFound issues a warning and prints its argument in tt
\def\ESHpvNotFound#1{\PackageWarning{ESH}{\ESHpvNotFoundMessage{#1}}\texttt{#1}}

%% Defining substitutions

% ESHpvReadAndDefineSubstitution{#A}#B#C#B#D maps #A-\detokenize{#C} to #D
\def\ESHpvReadAndDefineSubstitution#1#2{%
  % \tmp reads everything up to the next separator (#2), restores catcodes
  % (\endgroup), and calls \ESHpvDefineSubstitution on the code it read.
  \def\tmp##1#2{\endgroup\ESHpvDefineSubstitution{#1}{##1}}\tmp}

% ESHpvDefineSubstitution{#A}{#B}#C maps #A-\detokenize{#B} to #C
\def\ESHpvDefineSubstitution#1#2{\expandafter\def\csname #1-\detokenize{#2}\endcsname}

%% Applying substitutions

% ESHpvReadAndSubstitute{#A}#B#C#B looks up #A-\detokenize{#C}
\def\ESHpvReadAndSubstitute#1#2{%
  % \tmp reads everything up to the next separator (#2), restores catcodes
  % (\endgroup), and calls \ESHpvSubstitute on the code it read.
  \def\tmp##1#2{\endgroup\ESHpvSubstitute{#1}{##1}}\tmp}

% \ESHpvSubstitute{#A}{#B} looks up #A-\detokenize{#B} and substitutes it
\def\ESHpvSubstitute#1#2{%
  \expandafter\ifx\csname #1-\detokenize{#2}\endcsname\relax
  \ESHpvNotFound{\detokenize{#2}}%
  \else
  \csname #1-\detokenize{#2}\expandafter\endcsname
  \fi}

%% High-level interface

\def\ESHpvLookup#1{\begingroup\ESHpvEnterVerbMode\ESHpvReadAndSubstitute{#1}}
\def\ESHpvDefine#1{\begingroup\ESHpvEnterVerbMode\ESHpvReadAndDefineSubstitution{#1}}
