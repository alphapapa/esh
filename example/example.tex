%%%%%%%%%%%%%%%%%%%%%%%%%%%%%%%%%%%%%%%%%%%%%%%%%%%%%%%%%%%%%%%%%
%       This file shows how to use and customize esh2tex.       %
% (Comments marked with a '*' indicate optional customizations) %
%%%%%%%%%%%%%%%%%%%%%%%%%%%%%%%%%%%%%%%%%%%%%%%%%%%%%%%%%%%%%%%%%

\documentclass{article}

%%%%%%%%%%%%%%%%%%%%%%%
% Load a few packages %
%%%%%%%%%%%%%%%%%%%%%%%

\usepackage[margin=0.8in]{geometry}
\usepackage{amsmath,amssymb}

%%%%%%%%%%%%%%%%%%%%%%%%%
% Load the ESH preamble %
%%%%%%%%%%%%%%%%%%%%%%%%%

%% ESH-preamble-here

%%%%%%%%%%%%%%%%%%%%%%
% *Load custom fonts %
%%%%%%%%%%%%%%%%%%%%%%

\makeatletter
\@ifundefined{XeTeXinterchartoks}{% Minimal pdfLaTeX setup
  \usepackage[T1]{fontenc}
  \usepackage[utf8]{inputenc}
  \usepackage{lmodern}
  \usepackage{MnSymbol}
}{% Regular XeLaTeX setup (recommended)
  \usepackage{fontspec}

  % Load XITS Math for symbols
  %%%%%%%%%%%%%%%%%%%%%%%%%%%%

  % https://github.com/khaledhosny/xits-math/
  \newfontfamily{\XITSMath}[Path=fonts/,
                            UprightFont=*.otf,
                            BoldFont=*bold.otf,
                            Mapping=tex-ansi]{xits-math}

  % Load Ubuntu Mono for code
  %%%%%%%%%%%%%%%%%%%%%%%%%%%

  % http://font.ubuntu.com/
  \newfontfamily{\UbuntuMono}[Path=fonts/,
                              UprightFont=*-R.ttf,
                              BoldFont=*-B.ttf,
                              ItalicFont=*-RI.ttf,
                              BoldItalicFont=*-BI.ttf,
                              Mapping=tex-ansi]{UbuntuMono}

  % Tell ESH about these fonts
  %%%%%%%%%%%%%%%%%%%%%%%%%%%%

  \renewcommand{\ESHFontFamily}{\UbuntuMono}
  \renewcommand{\ESHFallbackFontFamily}{\XITSMath}
}
\makeatother

%%%%%%%%%%%%%%%%%%%%%%
% *Customize spacing %
%%%%%%%%%%%%%%%%%%%%%%

\setlength{\ESHSkip}{0.8\baselineskip}

%%%%%%%%%%%%%%%%%%%%%%%%%%%
% *Customize environments %
%%%%%%%%%%%%%%%%%%%%%%%%%%%

% ESHInline
%%%%%%%%%%%

\newcommand{\angles}[1]{$\langle\,$#1$\,\rangle$}

\DeclareRobustCommand*{\ESHInline}[1]
  {{\ESHText{\ESHInlineInternalSetup\ESHInlineBasicSetup\angles{#1}}}}

% ESHBlock
%%%%%%%%%%

\renewenvironment{ESHBlock}{%
  \par\ESHNoBreakAddVSpace{\ESHSkip}\bgroup\ESHBlockInternalSetup\ESHBlockBasicSetup%
  \hrule\addvspace{0.5em}%
}{%
  \par\egroup\addvspace{0.5em}\hrule\addvspace{2\ESHSkip}%
}

%%%%%%%%%%%%%%%%%%%%%%%%%%%%%%%%%%%%%%%%%%%
% Define and register a few inline macros %
%%%%%%%%%%%%%%%%%%%%%%%%%%%%%%%%%%%%%%%%%%%

% (These macros are registered in esh-init.el)
% Make them all aliases of \verb to remain compatible with plain LaTeX

\def\cverb{\verb}
\def\javaverb{\verb}
\def\pythonverb{\verb}
\def\prettylisp{\verb}
\def\normallisp{\verb}

\begin{document}

\section*{C (source: \texttt{xfaces.c} in Emacs)}

%% ESH: c
\begin{verbatim}
#if defined HAVE_X_WINDOWS && defined USE_X_TOOLKIT

/* Make menus on frame F appear as specified by the `menu' face.  */
static void
x_update_menu_appearance (struct frame *f)
{
  struct x_display_info *dpyinfo = FRAME_DISPLAY_INFO (f);
  XrmDatabase rdb;

  if (dpyinfo && (rdb = XrmGetDatabase (FRAME_X_DISPLAY (f)), rdb != NULL))
    {
      char line[512];
      char *buf = line;
      ptrdiff_t bufsize = sizeof line;
      Lisp_Object lface = lface_from_face_name (f, Qmenu, true);
      struct face *face = FACE_FROM_ID (f, MENU_FACE_ID);
\end{verbatim}

\section*{Emacs lisp (source: \texttt{esh.el} in this library)}

%% ESH: emacs-lisp
\begin{verbatim}
(require 'seq)
(require 'color)
(require 'subr-x)

;;; Misc utils

(defun esh--normalize-color (color)
  "Return COLOR as a hex string."
  (upcase (if (= (aref color 0) ?#) color
            (apply #'color-rgb-to-hex (color-name-to-rgb color)))))

(defun esh--filter-cdr (val alist)
  "Remove conses in ALIST whose `cdr' is VAL."
  (seq-filter (lambda (pair) (not (eq (cdr pair) val))) alist))
\end{verbatim}

\section*{Python (source: \texttt{monospacifier.py})}

%% ESH: python
\begin{verbatim}
class AllowWideCharsGlyphScaler(GlyphScaler):
    def __init__(self, cell_width, avg_width):
        """Construct instance based on target CELL_WIDTH and source AVG_WIDTH."""
        GlyphScaler.__init__(self, cell_width)
        self.avg_width = avg_width

    def scale(self, glyph):
        if glyph.width > 0:
            new_width_in_cells = int(math.ceil(0.75 * glyph.width / self.avg_width))
            # if new_width_in_cells > 1:
            #     print("{} is {} cells wide ({} -> {})".format(...))
            GlyphScaler.set_width(glyph, new_width_in_cells * self.cell_width)
\end{verbatim}

\clearpage

\section*{Perl (source: YAGOpt)}

%% ESH: perl
\begin{verbatim}
#&getopt("f:bar") ||
#	die &usage("script","f:bar","oo","[files ...]");
sub getopt {
    local($_,$flag,$opt,$f,$r,@temp) = @_;
    @temp = split(/(.):/);
    while ($#temp >= $[) {
        $flag .= shift(@temp);
        $opt .= shift(@temp);
    }
    while ($_ = $ARGV[0], /^-(.)(.*)/ && shift(@ARGV)) {
        ($f,$r) = ($1,$2);
        last if $f eq '-';
        if (index($flag,$f) >= $[) {
            eval "\$opt_$f++;";
            $r =~ /^(.)(.*)/,redo if $r ne '';
\end{verbatim}

\section*{Ruby (source: \texttt{parser.rb} in Ruby's standard library)}

%% ESH: ruby
\begin{verbatim}
class NotWellFormedError < Error
  attr_reader :line, :element

  # Create a new NotWellFormedError for an error at +line+ in +element+.
  def initialize(line=nil, element=nil)
    message = "This is not well formed XML"
    if element or line
      message << "\nerror occurred"
      message << " in #{element}" if element
    end
    message << "\n#{yield}" if block_given?
    super(message)
\end{verbatim}

\section*{Misc}

\subsection*{Inline snippets and inline blocks}

ESH works inline as well:

\begin{itemize}
\item Here's some C and some Python code: \cverb|int main() { return 0; }|, \pythonverb/def method(self, x): yield x/
\item
  \begin{tabular}[t]{@{}r@{ }l}
    Some Elisp with prettification: & \prettylisp,(lambda (x y) (or (<= x y) (approx= (/+/ x y) 0))), \\
            without prettification: & \normallisp!(lambda (x y) (or (<= x y) (approx= (/+/ x y) 0)))!
  \end{tabular}
\item And finally a few inline blocks:
\fbox{% Using raggedright because verbatim breaks when used inline
%% ESHInlineBlock: python
\begin{raggedright}
def main():
    return 0
\end{raggedright}}
\fbox{%
%% ESHInlineBlock[t]: python
\begin{raggedright}
def main():
    return 0
\end{raggedright}}
\fbox{%
%% ESHInlineBlock[b]: python
\begin{raggedright}
def main():
    return 0
\end{raggedright}}
\end{itemize}

\subsection*{Line breaking}

ESH allows line breaks to happen within inline code snippets (here is an example: \javaverb|private static volatile int counter = 0|), but not in code blocks:

%% ESH: emacs-lisp
\begin{verbatim}
(defun esh--normalize-color (color) (upcase (if (= (aref color 0) ?#) color (apply #'color-rgb-to-hex (color-name-to-rgb color)))))
\end{verbatim}

%%%%%%%%%%%%%%%%%%%%%%%%%%%%%%%%%%%%%%%%%%%%%%
% The following requires additional packages %
%%%%%%%%%%%%%%%%%%%%%%%%%%%%%%%%%%%%%%%%%%%%%%

\section*{Highlighting with non-core Emacs packages}

The following examples all depend on externally developped packages, and thus
require that you run \texttt{cask install} to install these dependencies (Cask
is the Emacs Lisp equivalent of Python's virtualenvs).

\subsection*{Haskell (source: \texttt{Monoid.hs} in Haskell's standard library)}

%% ESH: haskell
\begin{verbatim}
-- | The dual of a 'Monoid', obtained by swapping the arguments of 'mappend'.
newtype Dual a = Dual { getDual :: a }
        deriving (Eq, Ord, Read, Show, Bounded, Generic, Generic1)

instance Monoid a => Monoid (Dual a) where
        mempty = Dual mempty
        Dual x `mappend` Dual y = Dual (y `mappend` x)

-- | The monoid of endomorphisms under composition.
newtype Endo a = Endo { appEndo :: a -> a }
               deriving (Generic)

instance Monoid (Endo a) where
        mempty = Endo id
        Endo f `mappend` Endo g = Endo (f . g)
\end{verbatim}

\subsection*{Racket (source: \texttt{misc.rkt} in Racket's standard library)}

%% ESH: racket
\begin{verbatim}
(define-syntax define-syntax-rule
  (lambda (stx)
    (let-values ([(err) (lambda (what . xs) (apply raise-syntax-error
                                              'define-syntax-rule what stx xs))])
      (syntax-case stx ()
        [(dr (name . pattern) template)
         (identifier? #'name)
         (syntax/loc stx
           (define-syntax name
             (lambda (user-stx)
               (syntax-case** dr #t user-stx () free-identifier=? #f
                 [(_ . pattern) (syntax-protect (syntax/loc user-stx template))]
                 [_ (pattern-failure user-stx 'pattern)]))))]
\end{verbatim}

\subsection*{OCaml (source: \texttt{genlex.ml} in OCaml's standard library)}

%% ESH: tuareg
\begin{verbatim}
(** The lexer **)
let make_lexer keywords =
  let kwd_table = Hashtbl.create 17 in
  List.iter (fun s -> Hashtbl.add kwd_table s (Kwd s)) keywords;
  let ident_or_keyword id =
    try Hashtbl.find kwd_table id with
      Not_found -> Ident id
  and keyword_or_error c =
    let s = String.make 1 c in
    try Hashtbl.find kwd_table s with
      Not_found -> raise (Stream.Error ("Illegal character " ^ s))
\end{verbatim}

\subsection*{Dafny (source: \texttt{DutchFlag.dfy} in Dafny's repo)}

%% ESH: dafny
\begin{verbatim}
method DutchFlag(a: array<Color>)
  requires a != null modifies a
  ensures forall i,j :: 0 <= i < j < a.Length  ==>  Ordered(a[i], a[j])
  ensures multiset(a[..]) == old(multiset(a[..]))
{
  var r, w, b := 0, 0, a.Length;
  while w != b
    invariant 0 <= r <= w <= b <= a.Length;
    invariant forall i :: 0 <= i < r  ==>  a[i] == Red
    invariant multiset(a[..]) == old(multiset(a[..]))
  {   match a[w]
        case Red =>
          a[r], a[w] := a[w], a[r];
          r, w := r + 1, w + 1;
\end{verbatim}

\subsection*{F* (source: \texttt{Handshake.fst} in miTLS)}

%% ESH: fstar
\begin{verbatim}
val processServerFinished: KeySchedule.ks -> HandshakeLog.log -> (hs_msg * bytes) -> ST (result bytes)
  (requires (fun h -> True))
  (ensures (fun h_0 i h_1 -> True))

let processServerFinished ks log (m, l) =
   match m with
   | Finished (f) ->
     let svd = KeySchedule.ks_client_12_server_finished ks in
     if (equalBytes svd f.fin_vd) then
        let _ = log @@ (Finished (f)) in
    Correct svd
     else Error (AD_decode_error, "Finished MAC did not verify")
   | _ -> Error (AD_decode_error, "Unexpected state")
\end{verbatim}

\subsection*{Coq (source: \texttt{ExtendedLemmas.v} in Fiat; requires a local Proof General setup)}

%% ESH: prettified-coq
\begin{verbatim}
Lemma ProgOk_Chomp_lemma :
  forall `{FacadeWrapper (Value av) A} (ev: Env av) (key: StringMap.key)
    (prog: Stmt) (tail1 tail2: A -> Telescope av) ex (v: A),
    key ∉ ex ->
    ({{ tail1 v }} prog {{ tail2 v }} ∪ {{ [key |> wrap v] :: ex }} // ev <->
     {{ [[`key ->> v as vv]]::tail1 vv }} prog {{ [[`key ->> v as vv]]::tail2 vv }} ∪ {{ ex }} // ev).
Proof.
  repeat match goal with
         | _ => tauto
         | _ => progress (intros || split)
         | [ H: ?a /\ ?b |- _ ] => destruct H
         | [ H: ?a ≲ Cons _ _ _ ∪ _ |- _ ] => learn (Cons_PushExt _ _ _ _ _ H)
         | [ H: ProgOk ?fmap _ _ ?t1 ?t2, H': _ ≲ ?t1 ∪ ?fmap |- _ ] => destruct (H _ H'); no_dup
         | [ H: RunsTo _ _ ?from ?to, H': forall st, RunsTo _ _ ?from st -> _ |- _ ] => specialize (H' _ H)
         | [ H: _ ≲ _ ∪ [_ |> _] :: _ |- _ ] => apply Cons_PopExt in H
         end.
Qed.
\end{verbatim}

\end{document}
