%%%%%%%%%%%%%%%%%%%%%%%%%%%%%%%%%%%%%%%%%%%%%%%%%%%%%%%%%%%%%%%%%
%       This file shows how to use and customize esh2tex.       %
% (Comments marked with a '*' indicate optional customizations) %
%%%%%%%%%%%%%%%%%%%%%%%%%%%%%%%%%%%%%%%%%%%%%%%%%%%%%%%%%%%%%%%%%

\documentclass{article}

%%%%%%%%%%%%%%%%%%%%%%%
% Load a few packages %
%%%%%%%%%%%%%%%%%%%%%%%

\usepackage[margin=0.85in]{geometry}

%%%%%%%%%%%%%%%%%%%%%%
% *Load custom fonts %
%%%%%%%%%%%%%%%%%%%%%%

\usepackage{fontspec}

% Load XITS Math (monospacified) for symbols
%%%%%%%%%%%%%%%%%%%%%%%%%%%%%%%%%%%%%%%%%%%%

% https://github.com/cpitclaudel/monospacifier
\newfontfamily{\XITSMath}[Path=fonts/,
                          UprightFont=*_monospacified_for_UbuntuMono.ttf,
                          BoldFont=*-Bold_monospacified_for_UbuntuMono.ttf,
                          Mapping=tex-ansi]{XITSMath}

% Load Ubuntu Mono for code
%%%%%%%%%%%%%%%%%%%%%%%%%%%

% http://font.ubuntu.com/
\newfontfamily{\UbuntuMono}[Path=fonts/,
                            UprightFont=*-R.ttf,
                            BoldFont=*-B.ttf,
                            ItalicFont=*-RI.ttf,
                            BoldItalicFont=*-BI.ttf,
                            Mapping=tex-ansi]{UbuntuMono}

% Tell ESH about these fonts
%%%%%%%%%%%%%%%%%%%%%%%%%%%%

\newcommand{\ESHFont}{\UbuntuMono}
\newcommand{\ESHInlineFont}{\UbuntuMono}
\newcommand{\ESHSpecialCharFont}{\XITSMath}

%%%%%%%%%%%%%%%%%%%%%%
% *Customize spacing %
%%%%%%%%%%%%%%%%%%%%%%

\newlength{\ESHSkip}
\setlength{\ESHSkip}{0.5\baselineskip}

%%%%%%%%%%%%%%%%%%%%%%%%%%%
% *Customize environments %
%%%%%%%%%%%%%%%%%%%%%%%%%%%

% \ESHInline
%%%%%%%%%%%%

% Note the extra pair of braces in the definition, which prevent fonts and
% settings from affecting subsequent text
\newcommand*{\ESHInline}[1]{{\ESHBasicSetup\ESHInlineFont\fbox{#1}}}

% \ESHBlock
%%%%%%%%%%%

\newenvironment{ESHBlock}{%
  \ESHBasicSetup\ESHFont\par\addvspace{\ESHSkip}\hrule\addvspace{0.5em}%
}{%
  \par\addvspace{0.5em}\hrule\addvspace{4\ESHSkip}
}

%%%%%%%%%%%%%%%%%%%%%%%%%
% Load the ESH preamble %
%%%%%%%%%%%%%%%%%%%%%%%%%

%% ESH-preamble-here

%%%%%%%%%%%%%%%%%%%%%%%%%%%%%%%%%%%%%%%%%%%%%%%%%
% Define and register a few inline environments %
%%%%%%%%%%%%%%%%%%%%%%%%%%%%%%%%%%%%%%%%%%%%%%%%%

\def\cverb{\verb}
\def\pythonverb{\verb}
\def\prettylisp{\verb}
\def\normallisp{\verb}

%% ESH-inline: c-mode \cverb|...|
%% ESH-inline: python-mode \pythonverb|...|
%% ESH-inline: emacs-lisp-mode \normallisp|...|
%% ESH-inline: pretty-emacs-lisp-mode \prettylisp|...|

\begin{document}

\section*{C (source: \texttt{xfaces.c} in Emacs)}

%% ESH: c-mode
\begin{verbatim}
#if defined HAVE_X_WINDOWS && defined USE_X_TOOLKIT

/* Make menus on frame F appear as specified by the `menu' face.  */

static void
x_update_menu_appearance (struct frame *f)
{
  struct x_display_info *dpyinfo = FRAME_DISPLAY_INFO (f);
  XrmDatabase rdb;

  if (dpyinfo
      && (rdb = XrmGetDatabase (FRAME_X_DISPLAY (f)),
      rdb != NULL))
    {
      char line[512];
      char *buf = line;
      ptrdiff_t bufsize = sizeof line;
      Lisp_Object lface = lface_from_face_name (f, Qmenu, true);
      struct face *face = FACE_FROM_ID (f, MENU_FACE_ID);
      const char *myname = SSDATA (Vx_resource_name);
      bool changed_p = false;
#ifdef USE_MOTIF
      const char *popup_path = "popup_menu";
#else
      const char *popup_path = "menu.popup";
#endif

      if (STRINGP (LFACE_FOREGROUND (lface)))
\end{verbatim}

\section*{Emacs lisp (source: \texttt{esh.el} in this library)}

%% ESH: emacs-lisp-mode
\begin{verbatim}
(require 'seq)
(require 'color)
(require 'subr-x)

;;; Misc utils

(defun esh--normalize-color (color)
  "Return COLOR as a hex string."
  (upcase (if (= (aref color 0) ?#) color
            (apply #'color-rgb-to-hex (color-name-to-rgb color)))))

(defun esh--filter-cdr (val alist)
  "Remove conses in ALIST whose `cdr' is VAL."
  (seq-filter (lambda (pair) (not (eq (cdr pair) val))) alist))

(defun esh--extract-props (props str)
  "Read PROPS from STR as an ALIST or (PROP . VAL)."
  (seq-map (lambda (prop)
             (cons prop (get-text-property 0 prop str)))
           props))
\end{verbatim}

\clearpage

\section*{Python (source: \texttt{monospacifier.py})}

%% ESH: python-mode
\begin{verbatim}
class GlyphScaler(object):
    def __init__(self, cell_width):
        self.cell_width = cell_width

    @staticmethod
    def set_width(glyph, width):
        delta = width - glyph.width
        glyph.left_side_bearing += delta / 2
        glyph.right_side_bearing += delta - glyph.left_side_bearing
        glyph.width = width

class AllowWideCharsGlyphScaler(GlyphScaler):
    def __init__(self, cell_width, avg_width):
        """Construct instance based on target CELL_WIDTH and source AVG_WIDTH."""
        GlyphScaler.__init__(self, cell_width)
        self.avg_width = avg_width

    def scale(self, glyph):
        if glyph.width > 0:
            new_width_in_cells = int(math.ceil(0.75 * glyph.width / self.avg_width))
            # if new_width_in_cells > 1:
            #     print("{} is {} cells wide ({} -> {})".format(
            #         glyph.glyphname, new_width_in_cells, self.cell_width, glyph.width))
            GlyphScaler.set_width(glyph, new_width_in_cells * self.cell_width)
\end{verbatim}

\section*{Perl (source: YAGOpt)}

%% ESH: perl-mode
\begin{verbatim}
#Yet Another Getopt Routine In Perl
# jgreely@cis.ohio-state.edu, 89/11/1
#usage:
#&getopt("f:bar") ||
#	die &usage("script","f:bar","oo","[files ...]");

sub getopt {
    local($_,$flag,$opt,$f,$r,@temp) = @_;
    @temp = split(/(.):/);
    while ($#temp >= $[) {
        $flag .= shift(@temp);
        $opt .= shift(@temp);
    }
    while ($_ = $ARGV[0], /^-(.)(.*)/ && shift(@ARGV)) {
        ($f,$r) = ($1,$2);
        last if $f eq '-';
        if (index($flag,$f) >= $[) {
            eval "\$opt_$f++;";
            $r =~ /^(.)(.*)/,redo if $r ne '';
        } elsif (index($opt,$f) >= $[) {
            $r = $r eq '' ? shift(@ARGV) : $r;
            eval "\$opt_$f = \$r;";
        } else {
\end{verbatim}

\clearpage

\section*{Misc inline snippets}

ESH works inline as well:

\begin{itemize}
\item Here's some C code: \cverb|int main() { return 0; }|
\item Some Python: \pythonverb|def method(self, x): yield x|
\item
  \begin{tabular}[t]{@{}r@{ }l}
    Some Elisp with prettification: & \prettylisp|(lambda (x y) (or (<= x y) (approx= (/+/ x y) 0)))| \\
            without prettification: & \normallisp|(lambda (x y) (or (<= x y) (approx= (/+/ x y) 0)))| \\
  \end{tabular}
\end{itemize}

%%%%%%%%%%%%%%%%%%%%%%%%%%%%%%%%%%%%%%%%%%%%%%
% The following requires additional packages %
%%%%%%%%%%%%%%%%%%%%%%%%%%%%%%%%%%%%%%%%%%%%%%

% \section*{Coq (source: \texttt{Between.v} in Coq's standard library)}
%
% %% ESH: coq-mode
% \begin{verbatim}
%
% Implicit Types k l p q r : nat.
%
% Section Between.
%   Variables P Q : nat -> Prop.
%
%   Inductive between k : nat -> Prop :=
%     | bet_emp : between k k
%     | bet_S : forall l, between k l -> P l -> between k (S l).
%
%   Hint Constructors between: arith v62.
%
%   Lemma bet_eq : forall k1 k2, k1 = k2 -> between k1 k2.
%   Proof.
%     induction 1; auto with arith.
%   Qed.
% \end{verbatim}

\end{document}