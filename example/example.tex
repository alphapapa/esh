%%%%%%%%%%%%%%%%%%%%%%%%%%%%%%%%%%%%%%%%%%%%%%%%%%%%%%%%%%%%%%%%%
%       This file shows how to use and customize esh2tex.       %
% (Comments marked with a '*' indicate optional customizations) %
%%%%%%%%%%%%%%%%%%%%%%%%%%%%%%%%%%%%%%%%%%%%%%%%%%%%%%%%%%%%%%%%%

\documentclass{article}

%%%%%%%%%%%%%%%%%%%%%%%
% Load a few packages %
%%%%%%%%%%%%%%%%%%%%%%%

\usepackage[margin=0.8in]{geometry}

%%%%%%%%%%%%%%%%%%%%%%
% *Load custom fonts %
%%%%%%%%%%%%%%%%%%%%%%

\makeatletter
\@ifundefined{XeTeXinterchartoks}{% Minimal pdfLaTeX setup
  \usepackage[T1]{fontenc}
  \usepackage[utf8]{inputenc}
  \usepackage{lmodern}
}{% Regular XeLaTeX setup (recommended)
  \usepackage{fontspec}

  % Load XITS Math for symbols
  %%%%%%%%%%%%%%%%%%%%%%%%%%%%

  % https://github.com/khaledhosny/xits-math/
  \newfontfamily{\XITSMath}[Path=fonts/,
                            UprightFont=*.otf,
                            BoldFont=*bold.otf,
                            Mapping=tex-ansi]{xits-math}

  % Load Ubuntu Mono for code
  %%%%%%%%%%%%%%%%%%%%%%%%%%%

  % http://font.ubuntu.com/
  \newfontfamily{\UbuntuMono}[Path=fonts/,
                              UprightFont=*-R.ttf,
                              BoldFont=*-B.ttf,
                              ItalicFont=*-RI.ttf,
                              BoldItalicFont=*-BI.ttf,
                              Mapping=tex-ansi]{UbuntuMono}

  % Tell ESH about these fonts
  %%%%%%%%%%%%%%%%%%%%%%%%%%%%

  \newcommand{\ESHFont}{\UbuntuMono}
  \newcommand{\ESHInlineFont}{\UbuntuMono}
  \newcommand{\ESHFallbackFont}{\XITSMath}
}
\makeatother

%%%%%%%%%%%%%%%%%%%%%%
% *Customize spacing %
%%%%%%%%%%%%%%%%%%%%%%

\newlength{\ESHSkip}
\setlength{\ESHSkip}{0.8\baselineskip}

%%%%%%%%%%%%%%%%%%%%%%%%%%%
% *Customize environments %
%%%%%%%%%%%%%%%%%%%%%%%%%%%

% \ESHInline
%%%%%%%%%%%%

% Note the extra pair of braces in the definition, which prevent fonts and
% settings from affecting subsequent text
\usepackage{amsmath,amssymb}
\DeclareRobustCommand*{\ESHInline}[1]{{\ESHInlineInternalSetup\ESHInlineBasicSetup$\langle\,$#1$\,\rangle$}}

% \ESHBlock
%%%%%%%%%%%

\newenvironment{ESHBlock}{%
  \par\addvspace{\ESHSkip}\ESHBlockInternalSetup\ESHBlockBasicSetup\hrule\addvspace{0.5em}%
}{%
  \par\addvspace{0.5em}\hrule\addvspace{2\ESHSkip}
}

%%%%%%%%%%%%%%%%%%%%%%%%%
% Load the ESH preamble %
%%%%%%%%%%%%%%%%%%%%%%%%%

% This file is automatically created by esh2tex
%%%%%%%%%%%%%%%%%%%%%%%%%%%
% DO NOT MODIFY THIS FILE %
%%%%%%%%%%%%%%%%%%%%%%%%%%%

%% Instead, redefine the relevant macros in your own source file, **after**
%% including this file.  This file is overwritten every time you run esh2tex
%% with the --write-preamble option.

%%%%%%%%%%%%%%
%% Core ESH %%
%%%%%%%%%%%%%%

\newcommand*{\ESHRequirePackage}[2][]
  {\IfFileExists{#2.sty}
    {\RequirePackage[#1]{#2}}
    {\PackageError{ESH}{Missing LaTeX dependency: please install #2}
      {ESH requires relsize, ulem, and xcolor to work properly.  On Debian derivatives, these packages are provided by texlive-recommended and texlive-latex-extra.}}}

% Packages
%%%%%%%%%%

\ESHRequirePackage{xcolor} % TeXLive-recommended
\ESHRequirePackage[normalem]{ulem} % TeXLive-recommended

\ESHRequirePackage{relsize} % TeXLive-latex-extra
\renewcommand{\RSsmallest}{1pt}
\renewcommand{\RSlargest}{50pt}

% Fonts
%%%%%%%

% Code blocks
\newcommand*{\ESHFontSize}{}
\newcommand*{\ESHFontFamily}{\ttfamily}
\newcommand*{\ESHFont}{\ESHFontSize\ESHFontFamily}

% Inline snippets
\newcommand*{\ESHInlineFontSize}{\ESHFontSize}
\newcommand*{\ESHInlineFontFamily}{\ESHFontFamily}
\newcommand*{\ESHInlineFont}{\ESHInlineFontSize\ESHInlineFontFamily}

% Inline blocks
\newcommand*{\ESHInlineBlockFontSize}{\ESHFontSize}
\newcommand*{\ESHInlineBlockFontFamily}{\ESHFontFamily}
\newcommand*{\ESHInlineBlockFont}{\ESHInlineBlockFontSize\ESHInlineBlockFontFamily}

% Blocks
\newcommand*{\ESHBlockFontSize}{\ESHFontSize}
\newcommand*{\ESHBlockFontFamily}{\ESHFontFamily}
\newcommand*{\ESHBlockFont}{\ESHBlockFontSize\ESHBlockFontFamily}

% Characters not covered by \ESHFont (XeTeX/LuaTeX only)
\newcommand*{\ESHFallbackFontFamily}{\ESHFontFamily}
\newcommand*{\ESHFallbackFont}{\ESHFallbackFontFamily}

% Utils
%%%%%%%

% \ESHNoHyphens disables hyphenation
\newcommand*{\ESHNoHyphens}{\hyphenpenalty=10000}

% \ESHConstantSpace ensures that spaces can't stretch
\newcommand*{\ESHConstantSpace}{\spaceskip=\fontdimen2\font\xspaceskip=0pt}

% \ESHCenterInWidthOf{#A}{#B} prints #B centered in a box as large as #A.
\newdimen\ESHtempdim%
\newcommand*{\ESHCenterInWidthOf}[2]
  {\settowidth\ESHtempdim{#1}%
   \makebox[\ESHtempdim][c]{#2}}

% ESHText switches out of math mode if needed (textnormal amstext-friendly)
\DeclareRobustCommand*{\ESHText}[1]{\ifmmode{\textnormal{#1}}\else{#1}\fi}

% \ESHIfFontChar{#A} uses a XeLaTeX/LuaTeX primitive to print #A in
% \ESHFallbackFont if it isn't covered by the current font.
\newcommand*{\ESHIfFontChar}[1]
  {\iffontchar\font`#1{#1}\else{\ESHFallbackFont#1}\fi}

\makeatletter
% In XeTeX and LuaTeX, \ESHWithFallback{#A} prints #A in the current font if
% possible and falls back to \ESHFallbackFont. In other engines, it always uses
% the fallback font.  The XeTeX implementation is derived from
% https://tug.org/pipermail/xetex/2011-November/022319.html.
\@ifundefined{XeTeXinterchartoks}
  {\@ifundefined{luatexversion}
     {% Not using XeTeX, nor LuaTeX: fall back to just using \ESHFallbackFont
      \def\ESHWithFallback#1{\ESHFallbackFont#1}}
     {% Using LuaTeX
      \def\ESHWithFallback#1{\ESHIfFontChar{#1}}}}
  {% Using XeTeX
   \def\ESHWithFallback#1{%
     \ifnum\XeTeXfonttype\font>0% Graphite, OpenType, or AAT font
       \ESHIfFontChar{#1}%
     \else% Legacy TeX font
       \setbox0=\hbox{\tracinglostchars=0\kern1sp#1\expandafter}%
       \ifnum\lastkern=1{\ESHFallbackFont#1}\else{#1}\fi
     \fi}}
\makeatother

% \ESH*SpecialChar is used to wrap non-ascii characters, which may need a fallback font
\DeclareRobustCommand*{\ESHInlineSpecialChar}[1]
  {{\ESHInlineFontFamily\ESHWithFallback{#1}}}
\DeclareRobustCommand*{\ESHBlockSpecialChar}[1]
  {{\ESHCenterInWidthOf{\ESHBlockFontFamily{a}}{\ESHBlockFontFamily\ESHWithFallback{#1}}}}
% \ESH*UnicodeSubstitution is for special characters replaced by equivalent math commands
\DeclareRobustCommand*{\ESHInlineUnicodeSubstitution}[1]
  {{\ESHInlineFontFamily#1}}
\DeclareRobustCommand*{\ESHBlockUnicodeSubstitution}[1]
  {{\ESHCenterInWidthOf{\ESHBlockFontFamily{a}}{\ESHBlockFontFamily#1}}}
\DeclareRobustCommand*{\ESHMathSymbol}[1]{\ensuremath{#1}}

% \ESHSetCurFontSize sets \ESHCurFontSize to the current font size (1 CSS em)
\makeatletter
\newlength{\ESHCurFontSize}
\newcommand*{\ESHSetCurFontSize}{\setlength{\ESHCurFontSize}{\f@size pt}}
\makeatother

% \ESH*Raise implements sub/superscripts
\DeclareRobustCommand*{\ESHInlineRaise}[2]
  {\ESHSetCurFontSize\raisebox{#1\ESHCurFontSize}{\relsize{-2}#2}}
\DeclareRobustCommand*{\ESHBlockRaise}[2]
  {\rlap{\ESHInlineRaise{#1}{#2}}\hphantom{#2}}

% ESH*Strut implements struts
\newlength{\ESHBaselineskip}
\DeclareRobustCommand*{\ESHBlockStrut}[1]
  {\rule{0pt}{#1\ESHBaselineskip}}

\makeatletter
% ESHUnderline produces a straight underline
\DeclareRobustCommand*{\ESHUnderline}[1]
  {\bgroup\UL@setULdepth\markoverwith{#1\rule[-\ULdepth]{2pt}{0.4pt}}\ULon}
% ESHUnderwave produces a wavy underline
\DeclareRobustCommand*{\ESHUnderwave}[1]
  {\bgroup\UL@setULdepth\markoverwith{#1\raisebox{-\ULdepth}{\raisebox{-.5\height}{\ESHSmallWaveFont\char58}}}\ULon}
\font\ESHSmallWaveFont=lasyb10 scaled 400
\makeatother

% \ESHBoxSep is the vertical padding of \ESHBox
\newlength{\ESHBoxSep}
\setlength{\ESHBoxSep}{1pt}

% \ESHPhantomStrut{} adds a zero-width strut whose depth and height are that of "Ag"
\newcommand*{\ESHPhantomStrut}{\vphantom{Ag}}

% \ESHBox{#color}{#lineWidth}{#contents} wraps #contents in an border of width
% #lineWidth and of color #color.  The box has no horizontal padding, its
% vertical padding is determined by \ESHBoxSep, and it doesn't affect the height
% of the current line.
\newdimen\ESHBoxTempDim%
\newcommand*{\ESHBox}[3]
  {\setlength{\fboxsep}{\ESHBoxSep}%
   \setlength{\fboxrule}{#2}%
   \setlength{\ESHBoxTempDim}{\dimexpr-\fboxsep-\fboxrule\relax}%
   \ESHPhantomStrut{}\vphantom{#3}%
   \smash{\fbox{\hspace*{\ESHBoxTempDim}\ESHPhantomStrut{}#3\hspace*{\ESHBoxTempDim}}}}

% \ESHColorBox{#color}{#contents} adds a background of color #color to
% #contents.  The box has no horizontal padding, its vertical padding is
% determined by \ESHBoxSep, and it doesn't affect the height of the current
% line.
\newcommand*{\ESHColorBox}[2]
  {\setlength{\fboxsep}{\ESHBoxSep}%
   \setlength{\ESHBoxTempDim}{\dimexpr-\fboxsep-\fboxrule\relax}%
   \ESHPhantomStrut{}\vphantom{#2}%
   \smash{\colorbox[HTML]{#1}{\hspace*{\ESHBoxTempDim}\ESHPhantomStrut{}#2\hspace*{\ESHBoxTempDim}}}}

% \ESHWeight* and \ESHSlant* adjust weight and slant
\newcommand*{\ESHWeightLight}[1]{\textlf{#1}}
\newcommand*{\ESHWeightRegular}[1]{\textmd{#1}}
\newcommand*{\ESHWeightBold}[1]{\textbf{#1}}
\newcommand*{\ESHInlineSlantItalic}[1]{\textit{#1}}
\newcommand*{\ESHBlockSlantItalic}[1]{{\itshape{#1}}} % No italic correction
\newcommand*{\ESHInlineSlantOblique}[1]{\textsl{#1}}
\newcommand*{\ESHBlockSlantOblique}[1]{{\slshape{#1}}}
\newcommand*{\ESHSlantNormal}[1]{\textup{#1}}

% Environments and macros
%%%%%%%%%%%%%%%%%%%%%%%%%

\newcommand*{\ESHBreakingSpace}{\ }
\newcommand*{\ESHNonbreakingSpace}{\nobreakspace}

% Internal plumbing needed to make the same code work with Inline, Block, and
% InlineBlock.  See http://tex.stackexchange.com/questions/336936/ for details.
\let\ESHSpecialChar\ignorespaces%
\let\ESHUnicodeSubstitution\ignorespaces%
\let\ESHRaise\ignorespaces%
\let\ESHBol\ignorespaces%
\let\ESHEol\ignorespaces%
\let\ESHSpace\ignorespaces%
\let\ESHDash\ignorespaces%
\let\ESHSlantItalic\ignorespaces%
\DeclareRobustCommand*{\ESHInlineInternalSetup}
  {\def\ESHSpecialChar{\ESHInlineSpecialChar}\def\ESHUnicodeSubstitution{\ESHInlineUnicodeSubstitution}%
   \def\ESHRaise{\ESHInlineRaise}\def\ESHSlantItalic{\ESHInlineSlantItalic}%
   \def\ESHStrut{\relax}\def\ESHBol{\relax}\def\ESHEol{\newline}\def\ESHSpace{\ESHBreakingSpace}%
   \def\ESHDash{-}}
\DeclareRobustCommand*{\ESHInlineBlockInternalSetup}
  {\def\arraystretch{1}%
   \def\ESHSpecialChar{\ESHBlockSpecialChar}\def\ESHUnicodeSubstitution{\ESHBlockUnicodeSubstitution}%
   \def\ESHRaise{\ESHBlockRaise}\def\ESHSlantItalic{\ESHBlockSlantItalic}%
   \setlength{\ESHBaselineskip}{\baselineskip}\def\ESHStrut{\ESHBlockStrut}%
   \def\ESHBol{\-}\def\ESHEol{\cr}\def\ESHSpace{\ESHNonbreakingSpace}\def\ESHDash{\hbox{-}\nobreak}}
\DeclareRobustCommand*{\ESHBlockInternalSetup}
  {\def\ESHSpecialChar{\ESHBlockSpecialChar}\def\ESHUnicodeSubstitution{\ESHBlockUnicodeSubstitution}%
   \def\ESHRaise{\ESHBlockRaise}\def\ESHSlantItalic{\ESHBlockSlantItalic}%
   \setlength{\ESHBaselineskip}{\baselineskip}\def\ESHStrut{\ESHBlockStrut}%
   \def\ESHBol{\-}\def\ESHEol{\newline}\def\ESHSpace{\ESHNonbreakingSpace}\def\ESHDash{\hbox{-}\nobreak}}

% Basic setup used when entering each type of environment or macro
% See http://tex.stackexchange.com/questions/22852 for \leavevmode
% \ESHBlockBasicSetup used to set \parskip to 0, but \ESHEol isn't \par anymore
\newcommand*{\ESHInlineBasicSetup}
  {\leavevmode\ESHNoHyphens\ESHInlineFont}
\newcommand*{\ESHInlineBlockBasicSetup}
  {\ESHNoHyphens\ESHInlineBlockFont\ESHConstantSpace}
\newcommand*{\ESHBlockBasicSetup}
  {\setlength{\parindent}{0pt}\raggedright\ESHNoHyphens%
   \ESHBlockFont\ESHConstantSpace}

% ESH*Hook is used to inject code before every ESH segment
\newcommand*{\ESHHook}{}
\newcommand*{\ESHInlineHook}{\ESHHook}
\newcommand*{\ESHInlineBlockHook}{\ESHHook}
\newcommand*{\ESHBlockHook}{\ESHHook}

\makeatletter
% \ESHInline is used for inline code
\DeclareRobustCommand*{\ESHInline}[1]
  {\bgroup\ESHText{\ESHInlineInternalSetup\ESHInlineBasicSetup\ESHInlineHook#1}\egroup}

% \ESHInlineBlockVerticalAlignment is the default vertical alignment of inline blocks
\newcommand*{\ESHInlineBlockVerticalAlignment}{c}

% \ESHInlineBlock is used for inline code blocks
\newenvironment{ESHInlineBlock}[1][\ESHInlineBlockVerticalAlignment]
  {\bgroup\ESHInlineBlockInternalSetup\ESHInlineBlockBasicSetup\ESHInlineBlockHook\begin{tabular}[#1]{@{}l@{}}}
  {\end{tabular}\egroup}

% \ESHSkip is the amount to skip before and after an \ESHBlock
\newlength{\ESHSkip}
\setlength{\ESHSkip}{\baselineskip}

% \ESHNoBreakAddVSpace adds vertical space, but prevents a page break.
% \nobreak would use a \penalty (which breaks \addvspace), hence \addpenalty.
\newcommand*{\ESHNoBreakAddVSpace}[1]{\addpenalty{\@M}\addvspace{#1}}

% \ESHBlock is used for code blocks
\newenvironment{ESHBlock}
  {\par\ESHNoBreakAddVSpace{\ESHSkip}\bgroup\ESHBlockInternalSetup\ESHBlockBasicSetup\ESHBlockHook}
  {\par\egroup\addvspace{\ESHSkip}}
\makeatother

%% \input wrappers
%%%%%%%%%%%%%%%%%%

\DeclareRobustCommand*{\ESHInputInline}[1]{\ESHInline{\input{#1.esh.tex}\unskip}}
\DeclareRobustCommand*{\ESHInputInlineBlock}[2][\ESHInlineBlockVerticalAlignment]
  {\begin{ESHInlineBlock}[#1]\input{#2.esh.tex}\unskip\end{ESHInlineBlock}}
\DeclareRobustCommand*{\ESHInputBlock}[1]{\begin{ESHBlock}\input{#1.esh.tex}\unskip\end{ESHBlock}}

%%%%%%%%%%%%%%%%%%%%%%%%%%%%%%%%%%%%%%%%%%%
%% \ESHpv: ESH Precomputed-Verbs support %%
%%%%%%%%%%%%%%%%%%%%%%%%%%%%%%%%%%%%%%%%%%%

%% Adapted from https://tex.stackexchange.com/questions/336837/

%% Utilities

% \ESHpvEnterVerbMode sets all specials to catcode 12
% * \@makeother sets the catcode of its argument to 12 (other)
% * \dospecials applies \do to each special character
% (see http://tex.stackexchange.com/questions/12721/control-command-arguments)
\makeatletter
\def\ESHpvEnterVerbMode{\let\do\@makeother\dospecials}
\makeatother

% \ESHpvNotFound{#msg} issues a warning
\def\ESHpvNotFoundMessage#1{No highlighting found for "#1"; falling back to verbatim.}
\def\ESHpvNotFound#1{\PackageWarning{ESH}{\ESHpvNotFoundMessage{#1}}}

%% Defining substitutions

% ESHpvReadDelimitedAndDefineSubstitution{#lang}#delim#key#delim#value maps
% #lang-\detokenize{#key} to #value
\def\ESHpvReadDelimitedAndDefineSubstitution#1#2{%
  % \ESHpvInternalScanner reads tokens up to the next separator #2, restores
  % catcodes (\endgroup), and calls \ESHpvDefineSubstitution on these tokens
  \def\ESHpvInternalScanner##1#2{\endgroup\ESHpvDefineSubstitution{#1}{##1}}\ESHpvInternalScanner}

% ESHpvDefineSubstitution{#lang}{#key}#value maps #lang-\detokenize{#key} to #value
\def\ESHpvDefineSubstitution#1#2{\expandafter\def\csname #1-\detokenize{#2}\endcsname}

%% Applying substitutions

% \ESHpvSubstitute{#lang}{#key}{#msg}{#fallback} looks up a mapping for
% #lang-\detokenize{#key} and inserts it.  If no mapping can be found, it prints
% a warning based on #msg and includes #fallback in the document.
\def\ESHpvSubstitute#1#2#3#4{%
  \expandafter\ifx\csname #1-\detokenize{#2}\endcsname\relax
  \ESHpvNotFound{#3}\texttt{#4}%
  \else
  \csname #1-\detokenize{#2}\expandafter\endcsname
  \fi}

% \ESHpvSubstituteMacro{#lang}{#key} looks up #lang-\detokenize{#key} and
% substitutes it.  It falls back to #key itself if no mapping can be found.
\def\ESHpvSubstituteMacro#1#2{\ESHpvSubstitute{#1}{#2}{\detokenize{#2}}{\detokenize{#2}}}

% ESHpvReadDelimitedAndSubstitute{#lang}#delim#key#delim looks up
% #lang-\detokenize{#key} and calls \ESHpvSubstituteMacro.
\def\ESHpvReadDelimitedAndSubstitute#1#2{%
  % \ESHpvInternalScanner reads tokens up to the next separator #2, restores
  % catcodes (\endgroup), and calls \ESHpvSubstituteMacro on these tokens
  \def\ESHpvInternalScanner##1#2{\endgroup\ESHpvSubstituteMacro{#1}{##1}}\ESHpvInternalScanner}

%% High-level interface

% \ESHpvLookupVerb reads its argument like \verb
% Note that this won't work perfectly as an argument to a macro
\DeclareRobustCommand*{\ESHpvLookupVerb}[1]
  {\begingroup\ESHpvEnterVerbMode\ESHpvReadDelimitedAndSubstitute{macro-#1}}

% \ESHpvDefineVerb#lang creates a new [code → highlight] record in table macro-#A
\DeclareRobustCommand*{\ESHpvDefineVerb}[1]
  {\begingroup\ESHpvEnterVerbMode\ESHpvReadDelimitedAndDefineSubstitution{macro-#1}}

%% Convenience functions for wrapping verbs in boxes

\newcommand*{\ESHSaveBoxName}[1]{ESHSaveBox:#1}

\makeatletter
% \ESHEnsureSaveBox{#name} calls \newsavebox{#name} if needed.
\newcommand{\ESHEnsureSaveBox}[1]
  {\@ifundefined{#1}{\expandafter\newsavebox\csname#1\endcsname}{}}
\makeatother

% \begin{ESHSavedVerb}{#name}#def\end{ESHSavedVerb} creates a box called
% ESHSaveBox:#name, and saves #def into it.
\newenvironment*{ESHSavedVerb}[1]
  {\def\ESHCurBox{\ESHSaveBoxName{#1}}\ESHEnsureSaveBox{\ESHCurBox}\begin{lrbox}{0}\ignorespaces}
  {\unskip\end{lrbox}\global\expandafter\setbox\csname\ESHCurBox\endcsname=\box0\relax}

% \ESHUseVerb{#name} retrieves the contents of the box ESHSaveBox:#name
\newcommand{\ESHUseVerb}[1]{\expandafter\usebox\csname\ESHSaveBoxName{#1}\endcsname}


%%%%%%%%%%%%%%%%%%%%%%%%%%%%%%%%%%%%%%%%%%%
% Define and register a few inline macros %
%%%%%%%%%%%%%%%%%%%%%%%%%%%%%%%%%%%%%%%%%%%

% (These macros are registered in esh-init.el)
% Make them all aliases of \verb to remain compatible with plain LaTeX

\def\cverb{\verb}
\def\javaverb{\verb}
\def\pythonverb{\verb}
\def\prettylisp{\verb}
\def\normallisp{\verb}

\begin{document}

\section*{C (source: \texttt{xfaces.c} in Emacs)}

%% ESH: c
\begin{verbatim}
#if defined HAVE_X_WINDOWS && defined USE_X_TOOLKIT

/* Make menus on frame F appear as specified by the `menu' face.  */
static void
x_update_menu_appearance (struct frame *f)
{
  struct x_display_info *dpyinfo = FRAME_DISPLAY_INFO (f);
  XrmDatabase rdb;

  if (dpyinfo && (rdb = XrmGetDatabase (FRAME_X_DISPLAY (f)), rdb != NULL))
    {
      char line[512];
      char *buf = line;
      ptrdiff_t bufsize = sizeof line;
      Lisp_Object lface = lface_from_face_name (f, Qmenu, true);
      struct face *face = FACE_FROM_ID (f, MENU_FACE_ID);
\end{verbatim}

\section*{Emacs lisp (source: \texttt{esh.el} in this library)}

%% ESH: emacs-lisp
\begin{verbatim}
(require 'seq)
(require 'color)
(require 'subr-x)

;;; Misc utils

(defun esh--normalize-color (color)
  "Return COLOR as a hex string."
  (upcase (if (= (aref color 0) ?#) color
            (apply #'color-rgb-to-hex (color-name-to-rgb color)))))

(defun esh--filter-cdr (val alist)
  "Remove conses in ALIST whose `cdr' is VAL."
  (seq-filter (lambda (pair) (not (eq (cdr pair) val))) alist))
\end{verbatim}

\section*{Python (source: \texttt{monospacifier.py})}

%% ESH: python
\begin{verbatim}
class AllowWideCharsGlyphScaler(GlyphScaler):
    def __init__(self, cell_width, avg_width):
        """Construct instance based on target CELL_WIDTH and source AVG_WIDTH."""
        GlyphScaler.__init__(self, cell_width)
        self.avg_width = avg_width

    def scale(self, glyph):
        if glyph.width > 0:
            new_width_in_cells = int(math.ceil(0.75 * glyph.width / self.avg_width))
            # if new_width_in_cells > 1:
            #     print("{} is {} cells wide ({} -> {})".format(...))
            GlyphScaler.set_width(glyph, new_width_in_cells * self.cell_width)
\end{verbatim}

\clearpage

\section*{Perl (source: YAGOpt)}

%% ESH: perl
\begin{verbatim}
#&getopt("f:bar") ||
#	die &usage("script","f:bar","oo","[files ...]");
sub getopt {
    local($_,$flag,$opt,$f,$r,@temp) = @_;
    @temp = split(/(.):/);
    while ($#temp >= $[) {
        $flag .= shift(@temp);
        $opt .= shift(@temp);
    }
    while ($_ = $ARGV[0], /^-(.)(.*)/ && shift(@ARGV)) {
        ($f,$r) = ($1,$2);
        last if $f eq '-';
        if (index($flag,$f) >= $[) {
            eval "\$opt_$f++;";
            $r =~ /^(.)(.*)/,redo if $r ne '';
\end{verbatim}

\section*{Ruby (source: \texttt{parser.rb} in Ruby's standard library)}

%% ESH: ruby
\begin{verbatim}
class NotWellFormedError < Error
  attr_reader :line, :element

  # Create a new NotWellFormedError for an error at +line+ in +element+.
  def initialize(line=nil, element=nil)
    message = "This is not well formed XML"
    if element or line
      message << "\nerror occurred"
      message << " in #{element}" if element
    end
    message << "\n#{yield}" if block_given?
    super(message)
\end{verbatim}

\section*{Misc}

\subsection*{Inline snippets and inline blocks}

ESH works inline as well:

\begin{itemize}
\item Here's some C and some Python code: \cverb|int main() { return 0; }|, \pythonverb/def method(self, x): yield x/
\item
  \begin{tabular}[t]{@{}r@{ }l}
    Some Elisp with prettification: & \prettylisp,(lambda (x y) (or (<= x y) (approx= (/+/ x y) 0))), \\
            without prettification: & \normallisp!(lambda (x y) (or (<= x y) (approx= (/+/ x y) 0)))!
  \end{tabular}
\item And finally an inline block: \fbox{% Using raggedright because verbatim breaks when used inline
%% ESHInlineBlock: python
\begin{raggedright}
def main():
    return 0
\end{raggedright}}
\end{itemize}

\subsection*{Line breaking}

ESH allows line breaks to happen within inline code snippets (here is an example: \javaverb|private static volatile int counter = 0|), but not in code blocks:

%% ESH: emacs-lisp
\begin{verbatim}
(defun esh--normalize-color (color) (upcase (if (= (aref color 0) ?#) color (apply #'color-rgb-to-hex (color-name-to-rgb color)))))
\end{verbatim}

%%%%%%%%%%%%%%%%%%%%%%%%%%%%%%%%%%%%%%%%%%%%%%
% The following requires additional packages %
%%%%%%%%%%%%%%%%%%%%%%%%%%%%%%%%%%%%%%%%%%%%%%

\section*{Highlighting with non-core Emacs packages}

The following examples all depend on externally developped packages, and thus
require that you run \texttt{cask install} to install these dependencies (Cask
is the Emacs Lisp equivalent of Python's virtualenvs).

\subsection*{Haskell (source: \texttt{Monoid.hs} in Haskell's standard library)}

%% ESH: haskell
\begin{verbatim}
-- | The dual of a 'Monoid', obtained by swapping the arguments of 'mappend'.
newtype Dual a = Dual { getDual :: a }
        deriving (Eq, Ord, Read, Show, Bounded, Generic, Generic1)

instance Monoid a => Monoid (Dual a) where
        mempty = Dual mempty
        Dual x `mappend` Dual y = Dual (y `mappend` x)

-- | The monoid of endomorphisms under composition.
newtype Endo a = Endo { appEndo :: a -> a }
               deriving (Generic)

instance Monoid (Endo a) where
        mempty = Endo id
        Endo f `mappend` Endo g = Endo (f . g)
\end{verbatim}

\subsection*{Racket (source: \texttt{misc.rkt} in Racket's standard library)}

%% ESH: racket
\begin{verbatim}
(define-syntax define-syntax-rule
  (lambda (stx)
    (let-values ([(err) (lambda (what . xs) (apply raise-syntax-error
                                              'define-syntax-rule what stx xs))])
      (syntax-case stx ()
        [(dr (name . pattern) template)
         (identifier? #'name)
         (syntax/loc stx
           (define-syntax name
             (lambda (user-stx)
               (syntax-case** dr #t user-stx () free-identifier=? #f
                 [(_ . pattern) (syntax-protect (syntax/loc user-stx template))]
                 [_ (pattern-failure user-stx 'pattern)]))))]
\end{verbatim}

\subsection*{OCaml (source: \texttt{genlex.ml} in OCaml's standard library)}

%% ESH: tuareg
\begin{verbatim}
(** The lexer **)
let make_lexer keywords =
  let kwd_table = Hashtbl.create 17 in
  List.iter (fun s -> Hashtbl.add kwd_table s (Kwd s)) keywords;
  let ident_or_keyword id =
    try Hashtbl.find kwd_table id with
      Not_found -> Ident id
  and keyword_or_error c =
    let s = String.make 1 c in
    try Hashtbl.find kwd_table s with
      Not_found -> raise (Stream.Error ("Illegal character " ^ s))
\end{verbatim}

\subsection*{Dafny (source: \texttt{DutchFlag.dfy} in Dafny's repo)}

%% ESH: dafny
\begin{verbatim}
method DutchFlag(a: array<Color>)
  requires a != null modifies a
  ensures forall i,j :: 0 <= i < j < a.Length  ==>  Ordered(a[i], a[j])
  ensures multiset(a[..]) == old(multiset(a[..]))
{
  var r, w, b := 0, 0, a.Length;
  while w != b
    invariant 0 <= r <= w <= b <= a.Length;
    invariant forall i :: 0 <= i < r  ==>  a[i] == Red
    invariant multiset(a[..]) == old(multiset(a[..]))
  {   match a[w]
        case Red =>
          a[r], a[w] := a[w], a[r];
          r, w := r + 1, w + 1;
\end{verbatim}

\subsection*{F* (source: \texttt{Handshake.fst} in miTLS)}

%% ESH: fstar
\begin{verbatim}
val processServerFinished: KeySchedule.ks -> HandshakeLog.log -> (hs_msg * bytes) -> ST (result bytes)
  (requires (fun h -> True))
  (ensures (fun h_0 i h_1 -> True))

let processServerFinished ks log (m, l) =
   match m with
   | Finished (f) ->
     let svd = KeySchedule.ks_client_12_server_finished ks in
     if (equalBytes svd f.fin_vd) then
        let _ = log @@ (Finished (f)) in
    Correct svd
     else Error (AD_decode_error, "Finished MAC did not verify")
   | _ -> Error (AD_decode_error, "Unexpected state")
\end{verbatim}

\subsection*{Coq (source: \texttt{ExtendedLemmas.v} in Fiat; requires a local Proof General setup)}

%% ESH: prettified-coq
\begin{verbatim}
Lemma ProgOk_Chomp_lemma :
  forall `{FacadeWrapper (Value av) A} (ev: Env av) (key: StringMap.key)
    (prog: Stmt) (tail1 tail2: A -> Telescope av) ex (v: A),
    key ∉ ex ->
    ({{ tail1 v }} prog {{ tail2 v }} ∪ {{ [key |> wrap v] :: ex }} // ev <->
     {{ [[`key ->> v as vv]]::tail1 vv }} prog {{ [[`key ->> v as vv]]::tail2 vv }} ∪ {{ ex }} // ev).
Proof.
  repeat match goal with
         | _ => tauto
         | _ => progress (intros || split)
         | [ H: ?a /\ ?b |- _ ] => destruct H
         | [ H: ?a ≲ Cons _ _ _ ∪ _ |- _ ] => learn (Cons_PushExt _ _ _ _ _ H)
         | [ H: ProgOk ?fmap _ _ ?t1 ?t2, H': _ ≲ ?t1 ∪ ?fmap |- _ ] => destruct (H _ H'); no_dup
         | [ H: RunsTo _ _ ?from ?to, H': forall st, RunsTo _ _ ?from st -> _ |- _ ] => specialize (H' _ H)
         | [ H: _ ≲ _ ∪ [_ |> _] :: _ |- _ ] => apply Cons_PopExt in H
         end.
Qed.
\end{verbatim}

\end{document}
